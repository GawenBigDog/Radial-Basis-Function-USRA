\documentclass[]{article}

\usepackage{amsmath}
%opening
\title{}
\author{}

\begin{document}

\maketitle

\begin{abstract}
 The methods of interpolation for 1-dimensional data, e.g. Fourier and polynomial interpolation, often employ the same principle: that for a set of distinct data sites $\{x_i\}_{i=1}^n$ and their corresponding values $\{f\}_{i=1}^n$ we can choose a set of basis functions $\{\psi\}_{i=1}^n$ such that a linear combination of the basis functions, $s(x)=\sum_{i=1}^n\lambda_i\psi_i(x)$, can be found satisfying $s(x_i)=f_i$. For data in 1D, many choices of bases will guarantee a non-singular linear system, independent of the data sites. However, this is not guaranteed in higher dimensions. This problem can be bypassed by choosing basis functions which are dependent on the data: translates of a function that is radially symmetric about its center. This is referred to as the Radial Basis Function method. This talk will introduce the method, discuss its limitations, and illustrate how it can be implemented by undergraduates.



\end{abstract}

\section{}

\end{document}

  