\documentclass[12pt,t]{beamer}
\usepackage{graphicx}
\usepackage{amsmath,amssymb}
\setbeameroption{hide notes}
\setbeamertemplate{note page}[plain]

% get rid of junk
\usetheme{default}
\beamertemplatenavigationsymbolsempty
\hypersetup{pdfpagemode=UseNone} % don't show bookmarks on initial view

% font
%\usepackage{fontspec}
%\setsansfont{TeX Gyre Heros}
%\setbeamerfont{note page}{family*=pplx,size=\footnotesize} % Palatino for notes
% "TeX Gyre Heros can be used as a replacement for Helvetica"
% In Unix, unzip the following into ~/.fonts
% In Mac, unzip it, double-click the .otf files, and install using "FontBook"
%   http://www.gust.org.pl/projects/e-foundry/tex-gyre/heros/qhv2.004otf.zip

% named colors
\definecolor{offwhite}{RGB}{249,242,215}
\definecolor{foreground}{RGB}{255,255,255}
\definecolor{background}{RGB}{24,24,24}
\definecolor{title}{RGB}{107,174,214}
\definecolor{gray}{RGB}{155,155,155}
\definecolor{subtitle}{RGB}{102,255,204}
\definecolor{hilight}{RGB}{102,255,204}
\definecolor{vhilight}{RGB}{255,111,207}
\definecolor{lolight}{RGB}{155,155,155}
%\definecolor{green}{RGB}{125,250,125}

% use those colors
\setbeamercolor{titlelike}{fg=title}
\setbeamercolor{subtitle}{fg=subtitle}
\setbeamercolor{institute}{fg=gray}
\setbeamercolor{normal text}{fg=foreground,bg=background}
\setbeamercolor{item}{fg=foreground} % color of bullets
\setbeamercolor{subitem}{fg=gray}
\setbeamercolor{itemize/enumerate subbody}{fg=gray}
\setbeamertemplate{itemize subitem}{{\textendash}}
\setbeamerfont{itemize/enumerate subbody}{size=\footnotesize}
\setbeamerfont{itemize/enumerate subitem}{size=\footnotesize}

% page number
\setbeamertemplate{footline}{%
    \raisebox{5pt}{\makebox[\paperwidth]{\hfill\makebox[20pt]{\color{gray}
          \scriptsize\insertframenumber}}}\hspace*{5pt}}

% add a bit of space at the top of the notes page
\addtobeamertemplate{note page}{\setlength{\parskip}{12pt}}

% a few macros
\newcommand{\bi}{\begin{itemize}}
\newcommand{\ei}{\end{itemize}}
\newcommand{\ig}{\includegraphics}
\newcommand{\subt}[1]{{\footnotesize \color{subtitle} {#1}}}

% title info
\title{Introduction to Radial Basis Functions}
\subtitle{}
\author{\href{https://github.com/jessebett/}{Jesse Bettencourt}}
\institute{\href{}{McMaster University}}
\date{\href{jessebett@gmail.com}{\tt \scriptsize jessebett@gmail.com}
\\[-4pt]
\href{http://github.com/jessebett}{\tt \scriptsize github.com/jessebett}
}

\graphicspath{{./Images/}}


\begin{document}

% title slide
{
\setbeamertemplate{footline}{} % no page number here
\frame{
  \titlepage
  \note{}} } 

\begin{frame}{Motivation}
Given a set of measurments \subt{$\{f_i\}_{i=1}^N$}
taken at corresponding data sites \subt{$\{x_i\}_{i=1}^N$}
we want to find an interpolation function \subt{$s(x)$}
that infroms us on our system at locations different from our data sites.\\
\bigskip 

\subt{Examples of Data Sites and Measurments}
\begin{itemize}
\item[1D:] A series of temperature measurments over a time period
\item[2D:] Surface tempearture of a lake based on measurments collected at sample surface locations 
\item[3D:] Distribution of temperature within a lake
\item[n-D:] Machine learning, financial models, system optimization
\end{itemize}

\note{}
\end{frame}

\begin{frame}{What makes a good fit?}
We want $s(x)$ to be a good fit to our data, but what does this mean?
\bi
\item Interpolation: We want $s(x)$ to \subt{exactly match} our measurments at our data sites. \\ 
i.e. $s(x_i)=f_i$ $\forall i\in\{0 \mathellipsis N \}$
\item Approximation: We want $s(x)$ to \subt{closely match} our measurments at our data sites.\\
i.e. $s(x_i)\approx f_i$ $\forall i\in\{0 \mathellipsis N \}$ \\
e.g. with Least Squares 
\ei

For our purposes today, we will only consider interpolation.

\note{}
\end{frame}

\begin{frame}[c]{Our Problem, Restated}

\subt{Interpolation of Scattered Data}

Given data $(x_i,f_i)$, $i=1, \mathellipsis, N$, such that $x_i \in \mathbb{R}^n$, $y_i \in \mathbb{R}$, we want to find a continuous function $s(x)$ such that $s(x_i)=f_i$ $\forall i\in\{0 \mathellipsis N \}$


\note{}
\end{frame}

\begin{frame}{A Familiar Approach}
\subt{Convenient Assumtption}

Assume $s(x)$ is a linear combination of \subt{basis functions} $\psi_j$
\begin{center}
$s(x)=\sum_{j=1}^N \lambda_j \psi_j$
\end{center}

\subt{Interpolation as a Linear System}

Following this assumption we have a system of linear equations
\begin{center}
$A\boldsymbol{\lambda}=\boldsymbol{f}$
\end{center}
 where
 \bi
\item[A] is called the \subt{interpolation matrix} whose entries are given by\\
\begin{center}
$A_{ij}=\psi_j(x_i)$, $i,j= 1 \mathellipsis N$
\end{center}
\item[$\boldsymbol{\lambda}$] $=\left[ \lambda_1, \mathellipsis, \lambda_N \right]^T$
\item[$\boldsymbol{f}$] $=\left[ f_1, \mathellipsis, f_N \right]^T$
\ei

\note{}
\end{frame}

\begin{frame}{The Well-Posed Problem}
\begin{center}
$A\boldsymbol{\lambda}=\boldsymbol{f}$
\end{center}

Solving this linear system, thus finding $s(x)$, is only possible if the problem \subt{well-posed}, i.e., $\exists$ a unique soltuion. 
\bigskip

\subt{Result from introductory linear algebra:} 

The problem will be well-posed if and only if the interpolation matrix A is \subt{non-singular}, i.e., $\det(A)\neq0$.
\bigskip

\subt{Note:} The non-singularity of A will depend on our choice of basis functions, $\psi_{j=1}^N$

\note{}
\end{frame}

\begin{frame}{Easily Well-Posed in 1D}
In 1D, many choices of basis functions will gauentee a well-posed problem as long as the data-sites are distinct. 
\bigskip

\subt{Example}

We are familiar with \subt{polynomial interpolation}, interpolating from N data sites with a $(N-1)$-degree polynomial. 
\begin{center}
$\psi_{j=1}^N=\{1,x,x^2,x^3, \mathellipsis, x^{N-1}\}$
\end{center}
\note{}
\end{frame}

\begin{frame}{A Problem in Higher Dimensions}
For n-Dimensions where $n\geq 2$ there is no such gaurentee.
\bigskip

For any set of basis functions, $\psi_{j=1}^N$ (chosen independently of the data sites) $\exists$ a set of distinct data sites $\{x_i\}_{i=1}^N$
such that the interpolation matrix becomes singular. 
\bigskip

\subt{Implication:}
If we choose our basis functions independently of the data, we are not guarenteed a well-posed problem.
\bigskip

\subt{Note:}
This results from the Haar-Mairhuber-Curtis Theorem


\note{}
\end{frame}

\begin{frame}[c]{A Solution in Higher Dimensions}
\subt{Implication:}
If we choose our basis functions independently of the data, we are not guarenteed a well-posed problem.
\bigskip

\subt{Solution?}

Choose basis functions depending on the data!
\bigskip

\note{}
\end{frame}

\begin{frame}{Basis Functions Depending on Data}
First, consider what we call the \subt{basic function}\\
\begin{equation*}
\psi(x)=|x|
\end{equation*}

To produce our set of basis functions, we take translates of the basic function.

\begin{equation*}
\psi_j(x)=|x-x_j|, j=1, \mathellipsis, N
\end{equation*}

So each basis function, $\psi_j(x)$, is our basic function shifted so that the \subt{center} or \subt{knot} is positioned on a data site, $x_j$.
\bigskip

\subt{Note:} It's possible to have other choices of centers, but in most implementations the centers coincide with data sites.

\note{}
\end{frame}

\begin{frame}{Radial Basis Functions}

\begin{equation*}
\psi_j(x)=|x-x_j|, j=1, \mathellipsis, N
\end{equation*}

Notice that $\psi_k(x)$ are radially symmetric about their centers, for this reason we call these functions \subt{Radial Basis Functions}.
\bigskip

Since the basis functions only depend on distance, the interpolation matrix becomes
\begin{equation*}
A=
\begin{bmatrix}
|x_1-x_1| & |x_1-x_2| & \cdots & |x_1-x_N|\\
|x_2-x_1| & |x_2-x_2|& \cdots & |x_2-x_N|\\
\vdots & \vdots & \ddots & \vdots\\
|x_N-x_1| & |x_N-x_2|& \cdots & |x_N-x_N|
\end{bmatrix}
\end{equation*}
called a \subt{distance matrix}.

\note{}
\end{frame}

\begin{frame}{The Distance Matrix}
Distance matricies have the property that, for Euclidean distances in any dimensions, the distance matrix is non-singular.
\bigskip

This means that our interpolation problem

\begin{equation*}
\begin{bmatrix}
||x_1-x_1|| & ||x_1-x_2|| & \cdots & ||x_1-x_N||\\
||x_2-x_1|| & ||x_2-x_2||& \cdots & ||x_2-x_N||\\
\vdots & \vdots & \ddots & \vdots\\
||x_N-x_1|| & ||x_N-x_2||& \cdots & ||x_N-x_N||
\end{bmatrix}
\begin{bmatrix}
\lambda_1\\
\lambda_2\\
\vdots\\
\lambda_N
\end{bmatrix}
=
\begin{bmatrix}
f_1\\
f_2\\
\vdots\\
f_N
\end{bmatrix}
\end{equation*}
is well-posed!
\bigskip

Our interpolant becomes 
\subt{$s(x)=\sum_{j=1}^N \lambda_j ||x-x_j|| $}

\note{}
\end{frame}

\end{document}