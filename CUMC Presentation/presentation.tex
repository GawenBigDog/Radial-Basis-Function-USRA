\documentclass[12pt,t]{beamer}
\usepackage{graphicx}
\setbeameroption{hide notes}
\setbeamertemplate{note page}[plain]

% get rid of junk
\usetheme{default}
\beamertemplatenavigationsymbolsempty
\hypersetup{pdfpagemode=UseNone} % don't show bookmarks on initial view

% font
%\usepackage{fontspec}
%\setsansfont{TeX Gyre Heros}
%\setbeamerfont{note page}{family*=pplx,size=\footnotesize} % Palatino for notes
% "TeX Gyre Heros can be used as a replacement for Helvetica"
% In Unix, unzip the following into ~/.fonts
% In Mac, unzip it, double-click the .otf files, and install using "FontBook"
%   http://www.gust.org.pl/projects/e-foundry/tex-gyre/heros/qhv2.004otf.zip

% named colors
\definecolor{offwhite}{RGB}{249,242,215}
\definecolor{foreground}{RGB}{255,255,255}
\definecolor{background}{RGB}{24,24,24}
\definecolor{title}{RGB}{107,174,214}
\definecolor{gray}{RGB}{155,155,155}
\definecolor{subtitle}{RGB}{102,255,204}
\definecolor{hilight}{RGB}{102,255,204}
\definecolor{vhilight}{RGB}{255,111,207}
\definecolor{lolight}{RGB}{155,155,155}
%\definecolor{green}{RGB}{125,250,125}

% use those colors
\setbeamercolor{titlelike}{fg=title}
\setbeamercolor{subtitle}{fg=subtitle}
\setbeamercolor{institute}{fg=gray}
\setbeamercolor{normal text}{fg=foreground,bg=background}
\setbeamercolor{item}{fg=foreground} % color of bullets
\setbeamercolor{subitem}{fg=gray}
\setbeamercolor{itemize/enumerate subbody}{fg=gray}
\setbeamertemplate{itemize subitem}{{\textendash}}
\setbeamerfont{itemize/enumerate subbody}{size=\footnotesize}
\setbeamerfont{itemize/enumerate subitem}{size=\footnotesize}

% page number
\setbeamertemplate{footline}{%
    \raisebox{5pt}{\makebox[\paperwidth]{\hfill\makebox[20pt]{\color{gray}
          \scriptsize\insertframenumber}}}\hspace*{5pt}}

% add a bit of space at the top of the notes page
\addtobeamertemplate{note page}{\setlength{\parskip}{12pt}}

% a few macros
\newcommand{\bi}{\begin{itemize}}
\newcommand{\ei}{\end{itemize}}
\newcommand{\ig}{\includegraphics}
\newcommand{\subt}[1]{{\footnotesize \color{subtitle} {#1}}}

% title info
\title{Introduction to Radial Basis Functions}
\subtitle{}
\author{\href{https://github.com/jessebett/}{Jesse Bettencourt}}
\institute{\href{}{McMaster University}}
\date{\href{jessebett@gmail.com}{\tt \scriptsize jessebett@gmail.com}
\\[-4pt]
\href{http://github.com/jessebett}{\tt \scriptsize github.com/jessebett}
}

\graphicspath{{./Images/}}


\begin{document}

% title slide
{
\setbeamertemplate{footline}{} % no page number here
\frame{
  \titlepage
  \note{}} } 

\begin{frame}{Motivation}
Given a set of measurments \subt{$\{f_i\}_{i=1}^N$}
taken at corresponding data sites \subt{$\{x_i\}_{i=1}^N$}
we want to find an interpolation function \subt{$s(x)$}
that infroms us on our system at locations different from our data sites.\\


\subt{Examples of Data Sites and Measurments}
\begin{itemize}
\item[1D:] A series of temperature measurments over a time period
\item[2D:] Surface tempearture of a lake based on measurments collected at sample surface locations 
\item[3D:] Distribution of temperature within a lake
\item[n-D:] Machine learning, financial models, system optimization
\end{itemize}

\note{}
\end{frame}

\begin{frame}{What makes a good fit?}
We want $s(x)$ to be a good fit to our data, but what does this mean?
\bi
\item Interpolation: We want $s(x)$ to \subt{exactly match} our measurments at our data sites. \\ 
i.e. $s(x_i)=f_i$ $\forall i\in\{0 \mathellipsis N \}$
\item Approximation: We want $s(x)$ to \subt{closely match} our measurments at our data sites.\\
i.e. $s(x_i)\approx f_i$ $\forall i\in\{0 \mathellipsis N \}$ \\
e.g. with Least Squares 
\ei

For our purposes today, we will only consider interpolation.

\note{}
\end{frame}




\end{document}